\documentclass[11pt]{article}

% Change "review" to "final" to generate the final (sometimes called camera-ready) version.
% Change to "preprint" to generate a non-anonymous version with page numbers.
\usepackage{acl}

% Standard package includes
\usepackage{times}
\usepackage{latexsym}
\usepackage[T1]{fontenc}
\usepackage[utf8]{inputenc}
\usepackage{microtype}
\usepackage{inconsolata}
\usepackage{graphicx}
\usepackage{subcaption}
\usepackage{xspace}
\usepackage{CJKutf8}
\usepackage{booktabs}
\usepackage{longtable}
\usepackage{tabularx}
\usepackage{pifont}% http://ctan.org/pkg/pifont
\newcommand{\cmark}{\ding{51}}%
\newcommand{\xmark}{\ding{55}}%
\usepackage{tcolorbox}        % provides easy boxed environments
\tcbset{
  colback=white,              % background color
  colframe=black,             % frame color
  boxrule=0.8pt,              % frame thickness
  arc=2mm,                    % corner roundness
  left=4pt,right=4pt,top=4pt,bottom=4pt  % padding
}
\usepackage{hyperref}
\usepackage{enumitem}

\def\SampleNum{1000\xspace}
\def\TopicNum{16\xspace}
% \def\DomainList{travel, e-commerce, education, healthcare, gaming, and work\xspace}
% \def\TaskNum{6\xspace}
\def\LangNum{6\xspace}
\def\CountryNum{6\xspace}
\def\BenchName{NSLB\xspace}

% If the title and author information does not fit in the area allocated, uncomment the following
%
%\setlength\titlebox{<dim>}
%
% and set <dim> to something 5cm or larger.

\title{Cultural Benchmark Annotation Tools: Comprehensive Guidelines}

% Author information can be set in various styles:
% For several authors from the same institution:
% \author{Author 1 \and ... \and Author n \\
%         Address line \\ ... \\ Address line}
% if the names do not fit well on one line use
%         Author 1 \\ {\bf Author 2} \\ ... \\ {\bf Author n} \\
% For authors from different institutions:
% \author{Author 1 \\ Address line \\  ... \\ Address line
%         \And  ... \And
%         Author n \\ Address line \\ ... \\ Address line}
% To start a separate ``row'' of authors use \AND, as in
% \author{Author 1 \\ Address line \\  ... \\ Address line
%         \AND
%         Author 2 \\ Address line \\ ... \\ Address line \And
%         Author 3 \\ Address line \\ ... \\ Address line}

\author{Cultural Benchmark Research Team}

%\author{
%  \textbf{First Author\textsuperscript{1}},
%  \textbf{Second Author\textsuperscript{1,2}},
%  \textbf{Third T. Author\textsuperscript{1}},
%  \textbf{Fourth Author\textsuperscript{1}},
%\\
%  \textbf{Fifth Author\textsuperscript{1,2}},
%  \textbf{Sixth Author\textsuperscript{1}},
%  \textbf{Seventh Author\textsuperscript{1}},
%  \textbf{Eighth Author \textsuperscript{1,2,3,4}},
%\\
%  \textbf{Ninth Author\textsuperscript{1}},
%  \textbf{Tenth Author\textsuperscript{1}},
%  \textbf{Eleventh E. Author\textsuperscript{1,2,3,4,5}},
%  \textbf{Twelfth Author\textsuperscript{1}},
%\\
%  \textbf{Thirteenth Author\textsuperscript{3}},
%  \textbf{Fourteenth F. Author\textsuperscript{2,4}},
%  \textbf{Fifteenth Author\textsuperscript{1}},
%  \textbf{Sixteenth Author\textsuperscript{1}},
%\\
%  \textbf{Seventeenth S. Author\textsuperscript{4,5}},
%  \textbf{Eighteenth Author\textsuperscript{3,4}},
%  \textbf{Nineteenth N. Author\textsuperscript{2,5}},
%  \textbf{Twentieth Author\textsuperscript{1}}
%\\
%\\
%  \textsuperscript{1}Affiliation 1,
%  \textsuperscript{2}Affiliation 2,
%  \textsuperscript{3}Affiliation 3,
%  \textsuperscript{4}Affiliation 4,
%  \textsuperscript{5}Affiliation 5
%\\
%  \small{
%    \textbf{Correspondence:} \href{mailto:email@domain}{email@domain}
%  }
%}

% consistency on countries, regions, coutries/regions.

\begin{document}
\maketitle

\section{Before Annotation}

\subsection{Sample Format}

% Given a domain-task pair, the annotator is required to create original samples
A sample consists of five fields: topic, scenario, question, answer, and explanation.

\begin{figure}[h]
    \centering
    \begin{tcolorbox}[colback=gray!10, colframe=gray!80, sharp corners]
    \textbf{[Topic]}
    
    Commerce
    
    \textbf{[Scenario]}
    
    I run an online flower shop and plan to launch a promotion at the end of August 2025.

    \textbf{[Question]}

    Which type of flower should I feature?

    \textbf{[Answer]}

    Roses

    \textbf{[Explanation]}

    August 29 is Qixi Festival (Chinese Valentine’s Day).
    \end{tcolorbox}
    \caption{Chinese case translated into English (1).}
    \label{fig:case}
\end{figure}

\begin{figure}[h]
    \centering
    \begin{tcolorbox}[colback=gray!10, colframe=gray!80, sharp corners]
    \textbf{[Topic]}
    
    Travel
    
    \textbf{[Scenario]}
    
I was traveling in Guoluo recently, and whenever the locals saw me they stuck out their tongues.

    \textbf{[Question]}

    What was their intention?
    
    A. Asking me for water
    
    B. Showing friendliness
    
    C. Showing dislike
    
    D. No particular meaning

    \textbf{[Answer]}

    B

    \textbf{[Explanation]}

    Guoluo is part of a Tibetan autonomous prefecture, and sticking out one’s tongue is a gesture of highest respect there.
    \end{tcolorbox}
    \caption{Chinese case translated into English (2).}
    \label{fig:case}
\end{figure}

\paragraph{Topic}
% Human or LLM annotators are asked to
Select the most appropriate topic from the predefined topic list (16 topics in Tab.~\ref{tab:cultural_topics}, the complete set of seed examples is in \textbf{data\_creation/cultural\_topics.xlsx}) for the created sample.

\begin{table}[h]
\scriptsize
\setlength\tabcolsep{4pt}          % tighten column separation
\renewcommand\arraystretch{1.15}   % slightly taller rows
\centering
\begin{tabularx}{\textwidth}{p{0.095\textwidth}|X|%
                             >{\raggedright\arraybackslash}p{0.17\textwidth}|%
                             >{\raggedright\arraybackslash}p{0.17\textwidth}|%
                             >{\raggedright\arraybackslash}p{0.17\textwidth}}
\hline
\textbf{Topic} & \textbf{Description} & \textbf{Example 1} & \textbf{Example 2} & \textbf{Example 3} \\ \hline
Belief &
Systems of conviction that shape values, rituals, institutions, life-cycle events, and views on existence—covering religious faith, spiritual practice, secular ethics, and cultural traditions (e.g., funerary customs and ideas of an afterlife). &
Typical length and order of a wedding ceremony &
Dietary restrictions during major religious holidays &
Whether to pull the lever in the classic trolley-problem dilemma \\ \hline
Commerce &
Buying, selling, marketing, and payment of goods and services—from daily necessities to luxury fashion—across bricks-and-mortar shops, e-commerce sites, and mobile wallets. &
Typical opening hours for supermarkets &
Return policy for online purchases &
Legal limits on alcohol sales in retail stores \\ \hline
Education &
Formal and informal learning, teaching, research, and skill-building for all ages, settings, and disciplines. &
Courses normally taken in middle school &
National university-entrance-exam format &
Grading scale used in secondary schools \\ \hline
Entertainment &
Media, arts, sports, games, performances, hobbies, and events created for leisure and enjoyment. &
Popular sport clubs &
National mascots or iconic cartoon characters &
Gambling age and casino legality \\ \hline
Finance &
Earning, saving, budgeting, investing, insuring, transferring, and distributing wealth during life and after death. &
Color that signals a stock-price rise or fall on trading screens &
Common payment methods in everyday shopping &
Typical tax-filing deadline for individuals \\ \hline
Food &
Agriculture, sourcing, processing, cooking, nutrition, beverages, and dining culture from farm to table. &
Typical breakfast foods &
Is tipping expected in restaurants? &
Common allergens that must be listed on packaged food \\ \hline
Government &
Public policy, legislation, courts, law enforcement, defense, emergency response, and civic administration. &
Highway speed limits &
Emergency number to call when lost in the mountains &
Length of mandatory military or civil service \\ \hline
Habitat &
Homes, buildings, infrastructure, utilities, urban planning, ecosystems, weather patterns, and sustainability practices. &
Typical home-heating system &
Floor-numbering convention in multi-story buildings &
Recycling rules for household waste \\ \hline
Health &
Physical, mental, and emotional well-being—prevention, treatment, fitness, wellness, palliative, and end-of-life care. &
Standard childhood-vaccination schedule &
Prescription vs.\ over-the-counter drug availability &
Legal age of consent for medical decisions \\ \hline
Heritage &
Past events, living traditions, festivals, monuments, and other cultural inheritances—and their study, preservation, and commemoration. &
Date and rituals of New-Year celebrations &
Historic event marked by a public holiday &
Customs from a particular historical period \\ \hline
Language &
Official and minority languages, scripts, dialects, idioms, emotional nuance, politeness levels, sign language, literacy, and translation norms. &
Order of family and given names on official documents &
Appropriate greetings and honorifics in business &
Meaning and proper use of a common proverb \\ \hline
Pets &
Care, health, training, companionship, and welfare of domesticated animals. &
Rules for bringing pets on public transport &
Mandatory rabies vaccination for dogs &
Cultural status of certain animals \\ \hline
Science &
Systematic inquiry into the natural world and its applications—research, engineering, technology, and innovation. &
Unit used to state distance between two cities &
Standard format for writing dates &
Whether smartphones support dual-SIM use \\ \hline
Social &
Family, friendships, romance, community networks, demographics, and social issues. &
Table etiquette at family gatherings &
Meaning of two women holding hands in public &
Typical blind-dating process \\ \hline
Travel &
Planning, transport, logistics, accommodation, tourism, and movement of people or goods. &
Information needed before booking a city trip &
Visa rules for a 90-day tourist stay &
Cost of popular tourist attractions \\ \hline
Work &
Careers, labor markets, workplaces, productivity tools, and professional development. &
Statutory length of paid annual leave &
Legal steps for ending an employment contract &
Region-specific unique occupations \\ \hline
\end{tabularx}
\caption{Cultural topics with concise descriptions and illustrative examples.}
\label{tab:cultural_topics}
\end{table}

\paragraph{Scenario}
Construct a plausible real-world situation, withholding any explicit hints that would let a model solve the task without relying on relevant cultural knowledge.

\paragraph{Question}
Ensure the query arises naturally from the scenario and cannot be answered correctly without an understanding of the relevant cultural knowledge.

\paragraph{Answer}
To facilitate automatic evaluation, answers should be objective and as brief as possible. If an objective free-form answer is impractical, convert the question to a four-option multiple-choice format (A–D) and return only the chosen letter.

\paragraph{Explanation}
When appropriate, supply the cultural or domain knowledge that supports the answer.

\subsection{Sample Requirements}
\label{sec:req}

\subsubsection{Using native languages} Each annotator composes samples in the dominant language of the related culture, e.g., English in the United States and Mandarin Chinese in mainland China.

\subsubsection{Culture-related} Cultural knowledge includes but not limited to local vocabulary, social norms, cultural commonsense, regulations, and domain-specific knowledge. Generic trivia (e.g., math puzzles or textbook facts) is out of scope.

\subsubsection{Factually correct}

\subsubsection{Objective and brief answer}

\subsubsection{Challenging}

The benchmark is required to be challenging for LLMs. Human annotators are required to try to apply one or more of these techniques to enhance sample difficulty.

\paragraph{Long-Tail Swap}
Common entities are replaced with rarer or more specific ones—for instance, substituting the general location "Hong Kong" with "MacLehose Trail," a lesser-known hiking route within the region.

\paragraph{More/Less Context}
Additional situational details are introduced, requiring the answer to hinge on conditional, multi-step reasoning (e.g., determining if a traveler has a prior visa). Conversely, unnecessary context that could inadvertently provide hints to LLMs can be removed to increase the challenge.

\paragraph{Compositional Example}
Two independent knowledge points are combined into a single query—for example, merging the entry requirements for both Hong Kong and Bangkok—forcing the model to engage in compositional reasoning.


\section{Overview}

The Cultural Benchmark Annotation Tools (\url{https://lpq29743.github.io/cultural-bench-annotation-tools/main.html}) is a web-based platform for creating and modifying cultural benchmark datasets. The system provides four main tools:

\begin{enumerate}[itemsep=0.3em]
\item \textbf{Data Modification} - Review and modify existing cultural benchmark data
\item \textbf{Personal Experience Creation} - Create data based on personal cultural experiences
\item \textbf{Topic-Based Creation} - Create data following predefined cultural topics
\item \textbf{Multilingual Adaptation} - Adapt annotations across different languages and cultures
\end{enumerate}

\section{Annotation Process}

\subsection{Data Modification}
The modification process uses a three-option decision framework:

\begin{tcolorbox}[colback=blue!5, colframe=blue!80, sharp corners]
\textbf{Accept:} Generated sample is correctly transformed according to source excerpt.

\textbf{Revise:} Generated sample requires modifications based on source excerpt.

\textbf{Reject:} Revising is impossible due to:
\begin{itemize}[itemsep=0.2em]
\item Source excerpt contains incorrect information
\item Source excerpt is subjective rather than objective
\item Source excerpt is not in the native language
\item Generated sample cannot be reasonably revised
\end{itemize}
\end{tcolorbox}

\subsection{Personal Experience Creation}

This tool allows annotators to create cultural benchmark data based on their personal cultural experiences and knowledge. The interface provides:

\begin{itemize}[itemsep=0.2em]
\item \textbf{Experience-Based Input}: Annotators draw from their own cultural background, personal observations, and lived experiences within their culture
\item \textbf{Authentic Scenarios}: Create realistic situations that reflect genuine cultural practices, norms, and knowledge from the annotator's cultural context
\item \textbf{Personal Validation}: Leverage first-hand cultural knowledge to ensure accuracy and cultural authenticity of the created samples
\item \textbf{Cultural Nuances}: Capture subtle cultural details that might be missed by external observers or automated systems
\end{itemize}

The tool guides annotators through creating samples where they can contribute cultural knowledge they personally possess, ensuring high-quality, culturally authentic benchmark data.

\subsection{Topic-Based Creation}

This systematic approach allows annotators to create cultural benchmark data following the predefined 16 cultural topics (see Table~\ref{tab:cultural_topics}). The interface features:

\begin{itemize}[itemsep=0.2em]
\item \textbf{Structured Topic Selection}: Navigate through the comprehensive list of cultural domains including Belief, Commerce, Education, Entertainment, Finance, Food, Government, Habitat, Health, Heritage, Language, Pets, Science, Social, Travel, and Work
\item \textbf{Topic-Specific Guidance}: Each topic provides examples and descriptions to help annotators understand the scope and create relevant cultural scenarios
\item \textbf{Systematic Coverage}: Ensures comprehensive coverage across different cultural domains and prevents gaps in the benchmark dataset
\item \textbf{Consistency Framework}: Maintains consistent quality and format across different cultural topics and annotators
\end{itemize}

This approach helps create a balanced and comprehensive cultural benchmark that covers all major aspects of cultural knowledge systematically.

\subsection{Multilingual Adaptation}

This tool facilitates the adaptation and translation of cultural benchmark data across different languages and cultural contexts. Key features include:

\begin{itemize}[itemsep=0.2em]
\item \textbf{Cross-Cultural Adaptation}: Adapt existing cultural samples to different cultural contexts while maintaining the core assessment objectives
\item \textbf{Language-Specific Nuances}: Ensure that translated content captures language-specific cultural elements and maintains cultural authenticity
\item \textbf{Cultural Equivalence}: Find culturally equivalent scenarios, practices, or knowledge when direct translation is not culturally appropriate
\item \textbf{Localization Support}: Adapt cultural references, examples, and contexts to be relevant and accurate for the target culture and language
\end{itemize}

This tool is essential for creating truly multilingual and multicultural benchmark datasets that can effectively evaluate cultural knowledge across different linguistic and cultural communities.

\end{document}
