\documentclass[11pt]{article}

% Change "review" to "final" to generate the final (sometimes called camera-ready) version.
% Change to "preprint" to generate a non-anonymous version with page numbers.
\usepackage{acl}

% Standard package includes
\usepackage{times}
\usepackage{latexsym}
\usepackage[T1]{fontenc}
\usepackage[utf8]{inputenc}
\usepackage{microtype}
\usepackage{inconsolata}
\usepackage{graphicx}
\usepackage{subcaption}
\usepackage{xspace}
\usepackage{CJKutf8}
\usepackage{booktabs}
\usepackage{longtable}
\usepackage{tabularx}
\usepackage{pifont}% http://ctan.org/pkg/pifont
\newcommand{\cmark}{\ding{51}}%
\newcommand{\xmark}{\ding{55}}%
\usepackage{tcolorbox}        % provides easy boxed environments
\tcbset{
  colback=white,              % background color
  colframe=black,             % frame color
  boxrule=0.8pt,              % frame thickness
  arc=2mm,                    % corner roundness
  left=4pt,right=4pt,top=4pt,bottom=4pt  % padding
}

\def\SampleNum{1000\xspace}
\def\TopicNum{16\xspace}
% \def\DomainList{travel, e-commerce, education, healthcare, gaming, and work\xspace}
% \def\TaskNum{6\xspace}
\def\LangNum{6\xspace}
\def\CountryNum{6\xspace}
\def\BenchName{NSLB\xspace}

% If the title and author information does not fit in the area allocated, uncomment the following
%
%\setlength\titlebox{<dim>}
%
% and set <dim> to something 5cm or larger.

\title{Cultural Benchmark Annotation Guideline}

% Author information can be set in various styles:
% For several authors from the same institution:
% \author{Author 1 \and ... \and Author n \\
%         Address line \\ ... \\ Address line}
% if the names do not fit well on one line use
%         Author 1 \\ {\bf Author 2} \\ ... \\ {\bf Author n} \\
% For authors from different institutions:
% \author{Author 1 \\ Address line \\  ... \\ Address line
%         \And  ... \And
%         Author n \\ Address line \\ ... \\ Address line}
% To start a separate ``row'' of authors use \AND, as in
% \author{Author 1 \\ Address line \\  ... \\ Address line
%         \AND
%         Author 2 \\ Address line \\ ... \\ Address line \And
%         Author 3 \\ Address line \\ ... \\ Address line}

\author{XXX}

%\author{
%  \textbf{First Author\textsuperscript{1}},
%  \textbf{Second Author\textsuperscript{1,2}},
%  \textbf{Third T. Author\textsuperscript{1}},
%  \textbf{Fourth Author\textsuperscript{1}},
%\\
%  \textbf{Fifth Author\textsuperscript{1,2}},
%  \textbf{Sixth Author\textsuperscript{1}},
%  \textbf{Seventh Author\textsuperscript{1}},
%  \textbf{Eighth Author \textsuperscript{1,2,3,4}},
%\\
%  \textbf{Ninth Author\textsuperscript{1}},
%  \textbf{Tenth Author\textsuperscript{1}},
%  \textbf{Eleventh E. Author\textsuperscript{1,2,3,4,5}},
%  \textbf{Twelfth Author\textsuperscript{1}},
%\\
%  \textbf{Thirteenth Author\textsuperscript{3}},
%  \textbf{Fourteenth F. Author\textsuperscript{2,4}},
%  \textbf{Fifteenth Author\textsuperscript{1}},
%  \textbf{Sixteenth Author\textsuperscript{1}},
%\\
%  \textbf{Seventeenth S. Author\textsuperscript{4,5}},
%  \textbf{Eighteenth Author\textsuperscript{3,4}},
%  \textbf{Nineteenth N. Author\textsuperscript{2,5}},
%  \textbf{Twentieth Author\textsuperscript{1}}
%\\
%\\
%  \textsuperscript{1}Affiliation 1,
%  \textsuperscript{2}Affiliation 2,
%  \textsuperscript{3}Affiliation 3,
%  \textsuperscript{4}Affiliation 4,
%  \textsuperscript{5}Affiliation 5
%\\
%  \small{
%    \textbf{Correspondence:} \href{mailto:email@domain}{email@domain}
%  }
%}

% consistency on countries, regions, coutries/regions.

\begin{document}
\maketitle

\section{Before Annotation}

\subsection{Sample Format}

% Given a domain-task pair, the annotator is required to create original samples
A sample consists of five fields: topic, scenario, question, answer, and explanation.

\begin{figure}[h]
    \centering
    \begin{tcolorbox}[colback=gray!10, colframe=gray!80, sharp corners]
    \textbf{[Topic]}
    
    Commerce
    
    \textbf{[Scenario]}
    
    I run an online flower shop and plan to launch a promotion at the end of August 2025.

    \textbf{[Question]}

    Which type of flower should I feature?

    \textbf{[Answer]}

    Roses

    \textbf{[Explanation]}

    August 29 is Qixi Festival (Chinese Valentine’s Day).
    \end{tcolorbox}
    \caption{Chinese case translated into English (1).}
    \label{fig:case}
\end{figure}

\begin{figure}[h]
    \centering
    \begin{tcolorbox}[colback=gray!10, colframe=gray!80, sharp corners]
    \textbf{[Topic]}
    
    Travel
    
    \textbf{[Scenario]}
    
I was traveling in Guoluo recently, and whenever the locals saw me they stuck out their tongues.

    \textbf{[Question]}

    What was their intention?
    
    A. Asking me for water
    
    B. Showing friendliness
    
    C. Showing dislike
    
    D. No particular meaning

    \textbf{[Answer]}

    B

    \textbf{[Explanation]}

    Guoluo is part of a Tibetan autonomous prefecture, and sticking out one’s tongue is a gesture of highest respect there.
    \end{tcolorbox}
    \caption{Chinese case translated into English (2).}
    \label{fig:case}
\end{figure}

\paragraph{Topic}
% Human or LLM annotators are asked to
Select the most appropriate topic from the predefined topic list (16 topics in Tab.~\ref{tab:cultural_topics}, the complete set of seed examples is in \textbf{data\_creation/cultural\_topics.xlsx}) for the created sample.

\begin{table}[h]
\scriptsize
\setlength\tabcolsep{4pt}          % tighten column separation
\renewcommand\arraystretch{1.15}   % slightly taller rows
\centering
\begin{tabularx}{\textwidth}{p{0.095\textwidth}|X|%
                             >{\raggedright\arraybackslash}p{0.17\textwidth}|%
                             >{\raggedright\arraybackslash}p{0.17\textwidth}|%
                             >{\raggedright\arraybackslash}p{0.17\textwidth}}
\hline
\textbf{Topic} & \textbf{Description} & \textbf{Example 1} & \textbf{Example 2} & \textbf{Example 3} \\ \hline
Belief &
Systems of conviction that shape values, rituals, institutions, life-cycle events, and views on existence—covering religious faith, spiritual practice, secular ethics, and cultural traditions (e.g., funerary customs and ideas of an afterlife). &
Typical length and order of a wedding ceremony &
Dietary restrictions during major religious holidays &
Whether to pull the lever in the classic trolley-problem dilemma \\ \hline
Commerce &
Buying, selling, marketing, and payment of goods and services—from daily necessities to luxury fashion—across bricks-and-mortar shops, e-commerce sites, and mobile wallets. &
Typical opening hours for supermarkets &
Return policy for online purchases &
Legal limits on alcohol sales in retail stores \\ \hline
Education &
Formal and informal learning, teaching, research, and skill-building for all ages, settings, and disciplines. &
Courses normally taken in middle school &
National university-entrance-exam format &
Grading scale used in secondary schools \\ \hline
Entertainment &
Media, arts, sports, games, performances, hobbies, and events created for leisure and enjoyment. &
Popular sport clubs &
National mascots or iconic cartoon characters &
Gambling age and casino legality \\ \hline
Finance &
Earning, saving, budgeting, investing, insuring, transferring, and distributing wealth during life and after death. &
Color that signals a stock-price rise or fall on trading screens &
Common payment methods in everyday shopping &
Typical tax-filing deadline for individuals \\ \hline
Food &
Agriculture, sourcing, processing, cooking, nutrition, beverages, and dining culture from farm to table. &
Typical breakfast foods &
Is tipping expected in restaurants? &
Common allergens that must be listed on packaged food \\ \hline
Government &
Public policy, legislation, courts, law enforcement, defense, emergency response, and civic administration. &
Highway speed limits &
Emergency number to call when lost in the mountains &
Length of mandatory military or civil service \\ \hline
Habitat &
Homes, buildings, infrastructure, utilities, urban planning, ecosystems, weather patterns, and sustainability practices. &
Typical home-heating system &
Floor-numbering convention in multi-story buildings &
Recycling rules for household waste \\ \hline
Health &
Physical, mental, and emotional well-being—prevention, treatment, fitness, wellness, palliative, and end-of-life care. &
Standard childhood-vaccination schedule &
Prescription vs.\ over-the-counter drug availability &
Legal age of consent for medical decisions \\ \hline
Heritage &
Past events, living traditions, festivals, monuments, and other cultural inheritances—and their study, preservation, and commemoration. &
Date and rituals of New-Year celebrations &
Historic event marked by a public holiday &
Customs from a particular historical period \\ \hline
Language &
Official and minority languages, scripts, dialects, idioms, emotional nuance, politeness levels, sign language, literacy, and translation norms. &
Order of family and given names on official documents &
Appropriate greetings and honorifics in business &
Meaning and proper use of a common proverb \\ \hline
Pets &
Care, health, training, companionship, and welfare of domesticated animals. &
Rules for bringing pets on public transport &
Mandatory rabies vaccination for dogs &
Cultural status of certain animals \\ \hline
Science &
Systematic inquiry into the natural world and its applications—research, engineering, technology, and innovation. &
Unit used to state distance between two cities &
Standard format for writing dates &
Whether smartphones support dual-SIM use \\ \hline
Social &
Family, friendships, romance, community networks, demographics, and social issues. &
Table etiquette at family gatherings &
Meaning of two women holding hands in public &
Typical blind-dating process \\ \hline
Travel &
Planning, transport, logistics, accommodation, tourism, and movement of people or goods. &
Information needed before booking a city trip &
Visa rules for a 90-day tourist stay &
Cost of popular tourist attractions \\ \hline
Work &
Careers, labor markets, workplaces, productivity tools, and professional development. &
Statutory length of paid annual leave &
Legal steps for ending an employment contract &
Region-specific unique occupations \\ \hline
\end{tabularx}
\caption{Cultural topics with concise descriptions and illustrative examples.}
\label{tab:cultural_topics}
\end{table}

\paragraph{Scenario}
Construct a plausible real-world situation, withholding any explicit hints that would let a model solve the task without relying on relevant cultural knowledge.

\paragraph{Question}
Ensure the query arises naturally from the scenario and cannot be answered correctly without an understanding of the relevant cultural knowledge.

\paragraph{Answer}
To facilitate automatic evaluation, answers should be objective and as brief as possible. If an objective free-form answer is impractical, convert the question to a four-option multiple-choice format (A–D) and return only the chosen letter.

\paragraph{Explanation}
When appropriate, supply the cultural or domain knowledge that supports the answer.

\subsection{Sample Requirements}
\label{sec:req}

\subsubsection{Using native languages} Each annotator composes samples in the dominant language of the related culture, e.g., English in the United States and Mandarin Chinese in mainland China.

\subsubsection{Culture-related} Cultural knowledge includes but not limited to local vocabulary, social norms, cultural commonsense, regulations, and domain-specific knowledge. Generic trivia (e.g., math puzzles or textbook facts) is out of scope.

\subsubsection{Factually correct}

\subsubsection{Objective and brief answer}

\subsubsection{Challenging}

The benchmark is required to be challenging for LLMs. Human annotators are required to try to apply one or more of these techniques to enhance sample difficulty.

\paragraph{Long-Tail Swap}
Common entities are replaced with rarer or more specific ones—for instance, substituting the general location "Hong Kong" with "MacLehose Trail," a lesser-known hiking route within the region.

\paragraph{More/Less Context}
Additional situational details are introduced, requiring the answer to hinge on conditional, multi-step reasoning (e.g., determining if a traveler has a prior visa). Conversely, unnecessary context that could inadvertently provide hints to LLMs can be removed to increase the challenge.

\paragraph{Compositional Example}
Two independent knowledge points are combined into a single query—for example, merging the entry requirements for both Hong Kong and Bangkok—forcing the model to engage in compositional reasoning.

\section{Start Annotation}

\subsection{Data Modification}

You are provided with data generated by the LLM based on cultural datasets, following predefined schemas. The table below lists the important fields:

\begin{table}[h]
\centering
\resizebox{\textwidth}{!}{
\begin{tabular}{l|p{13cm}}
\hline
\textbf{Field} & \textbf{Description} \\ \hline
\texttt{source\_excerpt}  & Quoted passage from the original source, used as context for generation \\ \hline
\texttt{topic}            & One of the predefined \TopicNum\ cultural topics used in the dataset \\ \hline
\texttt{scenario}         & A narrative context in which the question is framed or asked \\ \hline
\texttt{question}         & The formulated question based on the scenario \\ \hline
\texttt{answer}           & The correct answer (answer key) to the question \\ \hline
\texttt{explanation}      & Concise justification or reasoning for the answer \\ \hline
\end{tabular}
}
\caption{Schema fields in the provided dataset. These enforce consistency and structure across all data items.}
\label{tab:metadata_fields}
\end{table}

\textbf{Annotation Instructions:}

\begin{enumerate}
    \item \textbf{File Identification:}
    \begin{itemize}
        \item Copy the \texttt{.xlsx} file you are working on to create a new instance for annotation.
        \item Append your ID to the filename to ensure traceability of the annotated document.
        \item For example, rename the file from \texttt{ar\_ma\_gpt4o} to \texttt{ar\_ma\_gpt4o\_linpq}.
    \end{itemize}
    
    \item \textbf{Annotation Tasks:}
    \begin{itemize}
        \item For each row in the \texttt{.xlsx} file, modify \textbf{COLUMN M} as part of your annotation, specifying whether to \textbf{Accept}, \textbf{Revise}, or \textbf{Reject} the item.
        \item If choosing \textbf{Revise}, update the relevant components (scenario, question, answer, explanation) to reflect the corrections based on \texttt{source\_excerpt}.
        \item Annotation options and their criteria:
        \begin{itemize}
            \item \textbf{Accept}: Select this option if the generated sample (scenario, question, answer, explanation) is correctly transformed according to \texttt{source\_excerpt}.
            \item \textbf{Revise}: Choose this option if the generated sample requires modifications based on \texttt{source\_excerpt}.
            \item \textbf{Reject}: Use this option if revising is impossible for the following reasons:
            \begin{itemize}
                \item \texttt{source\_excerpt} contains incorrect information.
                \item \texttt{source\_excerpt} is subjective and not objective.
                \item The generated sample or transformation cannot be reasonably revised.
            \end{itemize}
        \end{itemize}
    \end{itemize}
\end{enumerate}

% Copy the xlsx file your are working on and add your id to the xlsx to make sure we know this document is annotated, e.g., ar\_ma\_gpt4o to ar\_ma\_gpt4o\_linpq. For each row, your job is to \textbf{accept}, \textbf{revise}, or \textbf{reject} the item, by modifying COLUMN M.
% If the generated sample (scenario, question, answer, explanation) is correctly transformed by \texttt{source\_excerpt}, \textbf{accept} it. Otherwise try to \textbf{revise} the generated sample (scenario, question, answer, explanation) based on \texttt{source\_excerpt}. If impossible, e.g., \texttt{source\_excerpt} is wrong or not objective and is not possible to revise it, \textbf{reject} it.

\subsection{Data Creation}

Some important fields and their descriptions are listed below:

\begin{table}[h]
\centering
\resizebox{\textwidth}{!}{
\begin{tabular}{l|p{13cm}}
\hline
\textbf{Field} & \textbf{Description} \\ \hline
\texttt{topic}            & One of the \TopicNum\ predefined cultural topics \\ \hline
\texttt{scenario}         & A narrative context in which the question is framed or asked \\ \hline
\texttt{question}         & The formulated question corresponding to the scenario \\ \hline
\texttt{answer}           & The correct answer or answer key to the question \\ \hline
\texttt{explanation}      & A concise justification or reasoning for the answer \\ \hline
% \texttt{data\_creator}    & ID of the author (human or LLM) of the first draft \\ \hline
% \texttt{data\_verifier}   & ID of the human reviewer (must differ from \texttt{data\_creator}) \\ \hline
\end{tabular}
}
\caption{Metadata fields for each dataset item. These fields ensure consistency and adherence to predefined schemas.}
\label{tab:metadata_fields}
\end{table}

\vspace{0.5cm}

\noindent You can create the data inspired by the following sources:
\begin{itemize}
    \item \textbf{Personal experience:} Draw examples from your own perspective or cultural knowledge.
    \item \textbf{Local online forums:} Use credible forums or community discussions as inspiration.
    \item \textbf{Topic list with examples:} Refer to the predefined topics and examples provided in \texttt{data\_creation/cultural\_topics.xlsx}.
    \item \textbf{Multilingual data:} Leverage annotated data from other languages available in \texttt{data\_creation/annotated\_data.csv}.
\end{itemize}

\noindent When creating your own dataset:
\begin{itemize}
    \item Save the file as a \texttt{.xlsx} and name it using the format: \texttt{LanguageCode\_CountryCode\_Id.xlsx}. 
    \item For example: \texttt{zh\_cn\_linpq.xlsx}.
\end{itemize}

\noindent Keep in mind the following guidelines when creating samples:
\begin{itemize}
    \item Ensure the samples adhere to the specified requirements (\S\ref{sec:req}).
    \item Focus on creating challenging examples to test model performance effectively.
    \item It is not necessary to maintain a balanced distribution across topics.
\end{itemize}

% \bibliography{custom}

\end{document}
